\section{Latex Crash Course}
	Così io scrivo normalmente e posso fare un testo; per andare a capo semplicemente posso fare così \\ come potete vedere
	
	Lasciando una riga vuota di codice ho creato un nuovo paragrafo. Se voglio scrivere un parametro matematico in linea di testo basta che faccio $f_l(x) = 10N$, mentre se voglio un'equazione bella centrata:
	\begin{equation}
		f_{x, max} (t) = \int_{-\infty}^{\frac \pi 2} e^{-t^2} \, dt
	\end{equation}
	Se invece non voglio avere il numero per l'equazione basta fare 
	\[ f(x,y) = \cos(x) \sin(y) \]
	
	Adesso faccio un'equazione per usare i riferimenti:
	\begin{equation} \label{eq:esempioequazione}
		f(x) = e^x
	\end{equation}
	Quella appena scritta è l'equazione \ref{eq:esempioequazione} e si trova a pagina \pageref{eq:esempioequazione}.
	
	\paragraph{Elenchi puntati} Con questo comando posso creare un nuovo paragrafo con un certo titolo e farvi vedere come fare gli elenchi puntati:
	\begin{itemize}
		\item il primo punto;
		\item il secondo punto
	\end{itemize}
	L'elenco numerato si fa invece scrivendo
	\begin{enumerate}
		\item il primo punto;
		\item il secondo punto
	\end{enumerate}
	
	Per scrivere in corsivo basta fare \textit{così, usando il comando "textit"} mentre \textbf{per il grassetto utilizzo "textbf"}. Volendo \textbf{\textit{posso anche usarli insieme}}.




