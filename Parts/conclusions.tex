\section{Conclusions}
	The final robot is far from being finished, but this works provides a solid starting point to improve the actual machine design. At the current state it has not been fully decided the final model of the motor used to actuate the system, but that will determine just a slight variation in the design of the plate that's connecting the linear guides with the motor (that for this reason is not reported in the technical drawings).
	
	The designed assembly is also provided with a roller system running on the elevated track where the turret performing the operation can be built up on containing all the required tools.
	
	For what concerns the actuation of the system, we assumed that the machine will be controlled in open loop using stepper motors; for this reasons at the end of all tracks a touch sensor need to be mounted in order to provide the control system (that will be an \texttt{Arduino} micro controller) with a reset point to rely on. \\
	Such sensors can be positioned at one side of the ground tracks where a resting station should be designed, providing the turret a safe resting place in order to charge the on-board battery.
	
	\paragraph{Design improvements ideas} The proposed design still leaves room for further improvement. Considering as example the shipping of the components, it's not easy to move around a $4m$ long bar: for this reason a way to connect shorter T-slot profiles is a desired feature to improve component handling and transportability.