\section{Conclusions}
	The final robot is far from being finished, but this works provides a solid starting point to improve the actual machine design. At the current development state it has not been fully decided the final model of the motor used to actuate the system, however such decision will determine just a slight variation in the design of the plate used to connecting the motor to the plate of the linear guide; for this reason, the plate has not been reported in the technical drawings.
	
	The designed assembly is also provided with a roller system running on the elevated track where the turret performing the operation can be built up on containing all the required tools.
	
	For what concerns the actuation of the system, we assumed that the machine will be controlled in open loop using stepper motors; for this reasons at the end of all tracks a touch sensor need to be mounted in order to provide the control system (that will be an \texttt{Arduino} micro controller) with a reset point to rely on. \\
	Such sensors can be positioned at one side of the ground tracks where a resting station should be designed, providing the turret a safe place to charge the battery placed on board of the turret and that guarantees protection from external actions.
	
	\paragraph{Design improvements ideas} The proposed design still leaves room for further improvement.
	One thing that might be improved is the kit shipping capability: as we can see in the final bill of materials (appendix \ref{tab:BOM}, page \pageref{tab:BOM}) we have 3 main big components with a length greater then $3m$, making their transportation and handling difficult. One way to improve the design is by studying proper axial connections in order to join beams with smaller lengths while still not impacting the design of the gear-rack coupling and leaving proper clearances for the roller linear guides.
	
	