\section{Design process and verifications}
	With the data obtained from the preliminary analysis, the design has been carried using the 3D CAD software \texttt{AutoDesk Inventor Professional} with the help of the components library provided by the manufacturer Parker IPS. At this stage what we mainly did was to \textit{give shape} at the chosen concept by using mechanical elements provided by the library only in order to minimize the need of custom-made parts.
	
	For actuating the machine no off-the-shelf solution are available, hence we decided to create our custom rack design (\textbf{INSERIRE E CREARE RIFERIMENTO AL DISEGNO}): such component is made by a set of smaller bid that can be 3D printed with plastic material; such elements can be joined together by means of standard T-slot components. The rack is mounted on both tracks placed at the ground (motion in the $x$ direction) and the elevated one (motion in the $y$ direction); the gear attached to the motor that's actuating the system is a standard element made out of steel. Proper static verification of the component will be described in the following pages.
	
	Using the theory from the course we performed a static verification of the machine and we verified two critical bolted joints.
	
	\subsection{Static verification}
	With all the geometrical values known, it has been possible to carry out a more detailed analysis. Considering a more complex model of the forces that takes into account also the distributed load due to the gravity and that the force $F$ is applied with an eccentricity $e=5cm$, it has been possible to parametrically define the hyperstatic variables as
	\[ X_1 \simeq \big( 0.63-0.16\zeta\big) N\cdot m \qquad X_2 \simeq -8.10 \ N\cdot m \] \[ X_3 \simeq \big( - 10.54 - 11.07\zeta + 55.53\zeta^2 - 6.17\zeta^3 \big) N\cdot m \]
	Evaluating the stress state on the 3 beams making up the structure for values of $\zeta \in [0,L]$, what we obtained is that the most critical section is in the elevated track for $\zeta^* = 1.65m$ at $z^* = 1.65m$ where the maximum bending value $M_x^* \simeq 135.64N\cdot m$ is achieved. Knowing that $N^* = 0$ and neglecting the shear stress contribution due to both shear $V_y^* \simeq 111N$ and torque $M_z^* \simeq 1.89N\cdot m$, the stress state is
	\[ \sigma_{zz}(y) = \frac{M_x^*}{I_{xx,track}}y \qquad \Rightarrow \qquad \sigma_{zz,max} = \sigma_{zz}(2cm) \simeq 28.68 MPa   \]
	This leads to a safety factor of $\phi = \sigma_{ys}/\sigma_{zz,max} \simeq 8.37$, meaning that the structure is statically verified. This value is more then twice than the one achieved in the preliminary design phase but is due to the fact that in this case the full analytical solution of the hyperstatic problem has been considered, having determined all geometrical properties of the sections. As a reminder:
	\begin{itemize}
		\item due to the lack of analytical formulas for such complex geometry section, shear components due to torque and shears are neglected and could have influenced the mechanical system;
		\item we did not consider external forces acting that might act on the global $x$ axis due to the actuation of the machine or accidental load applied by the end user: this can lead to an increase in bending and torques that might make the system fail.
	\end{itemize}

	\textbf{MAGARI AGGIUNGERE GRAFICI DEL MOMENTO IN FUNZIONE DI 2/3 VALORI DI $\zeta$}
	
	\textbf{AGGIUNGERE LA STIFFNESS VERIFICATION}
	\subsection{Fatigue verification}
	\paragraph{Most critical section} As already stated, the most critical section is subjected to a maximum stress of $\sigma_{max} \simeq 28.68MPa$. While functioning we expect that the turret performing the operation to the soil periodically moves along the track: having $\zeta$ varying over time determines that also the internal loads are fluctuating and this might arise fatigue failure of the machine.
	
	Given the need of a fatigue verification, acknowledged that the minimum bending load at $z^*$ is $M_{x,min} \simeq 14.05N\cdot m$ achieved for $\zeta = 0m$, determining $\sigma_{min} \simeq 2.97MPa$; this further implies that the mean and amplitude stresses for fatigue verification are
	\[ \sigma_m \simeq 15.82 MPa \qquad \sigma_a \simeq 12.85MPa \]
	According to Soderberg criterion, the equivalent amplitude only stress component evaluates to
	\[ \sigma_{a,eq} = \frac{\sigma_{ys} \sigma_a}{\sigma_{ys} - \sigma_m} \simeq 13.76MPa  \] 
	In this case the load factor is $C_l=1$; given an equivalent diameter $d_{eq} = \sqrt{\frac 4\pi A} = 29mm$, the corresponding size factor is $C_d = 1.189d_{eq}^{-0.097} \simeq 0.86$; details concerning the surface finish of the T-slot beams are not enough to fully determine the surface coefficient $C_s$ that's assumed in this case to be $0.8$ as safety value. Considering an endurance limit of the 6061-T5 aluminium alloy of $\sigma_{lim} = 100MPa$ \cite{aluminium-endurance}, this means that the safety factor against fatigue failure is
	\[ \phi_\textrm{fatigue} = C_sC_dC_l\frac{\sigma_{lim}}{\sigma_{a,eq}} \simeq 4.98 \]