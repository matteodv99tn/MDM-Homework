\section{Design process and verifications}
	With the data obtained from the preliminary analysis, the design has been carried using the 3D CAD software \texttt{AutoDesk Inventor Professional} with the help of the components library provided by the manufacturer Parker IPS. At this stage what we mainly did was to \textit{give shape} at the chosen concept by using mechanical elements provided by the library only in order to minimize the need of custom-made parts.
	
	For actuating the machine no off-the-shelf solution are available, hence we decided to create our custom rack design (\textbf{INSERIRE E CREARE RIFERIMENTO AL DISEGNO}): such component is made by a set of smaller bid that can be 3D printed with plastic material; such elements can be joined together by means of standard T-slot components. The rack is mounted on both tracks placed at the ground (motion in the $x$ direction) and the elevated one (motion in the $y$ direction); the gear attached to the motor that's actuating the system is a standard element made out of steel. Proper static verification of the component will be described in the following pages.
	
	Using the theory from the course we performed a static verification of the machine and we verified two critical bolted joints.