\subsection{Design and static verification of rack and gear}
The verification continued with the parts responsible for actuating the machine, rack that is positioned on both tracks and gear attached on the motors' shaft.
We focused on two important aspects, interference and tooth root bending resistance.

\subsubsection*{Interference condition}
It's fundamental to check if this condition is fulfilled, because otherwise we could have a problem of undercut, when the tip of the rack cutter is moved beyond the base circle of the gear and the result is the weakening of the gear's teeth.

Interference condition is function of the minimum number of teeth of the pinion, in accordance with the following equation:
\begin{equation}
	\centering
	z_{min} = \frac{2}{sin^2\alpha},
\end{equation}
where $\alpha$ is the pressure angle, as we can see in figure \ref{fig:rack}.
\begin{SCfigure}[1][hbt]
	\centering
	\includegraphics[scale=1]{Images/rack.png}
	\caption{standard rack and pressure angle $\alpha$.}
	\label{fig:rack}
\end{SCfigure}
In our case we designed the profile with a pressure $\alpha = 20^\circ$, thus we have:
\begin{equation*}
	z_{min} = 17,
\end{equation*}

\subsubsection*{Gear design and verification}
	The gear attached to the motor's shaft is designed choosing as number of tooth $z=18$; assuming that the motors actuating the machines can generate a power $P=10W$ at $n = 50rpm$, the associated maximum generated torque is $T=1.9N\cdot m$. Choosing as module $m$ of the gear the value $2mm$, then the nominal diameter of the gear is $d = zm = 36mm$ and the maximum force that can be transmitted by the gear is
	\[ F = \frac T r = 2 \frac T d = 106N\]
	
	\paragraph{Tooth root fatigue resistance}	One of the most important problem in the usage of gears is the teeth fatigue resistance, due to the fact they're subjected to pulsating loads.  
	In our analysis we used the Lewis method that consider the tooth as a cantilever beam and shear and compressive normal stresses are neglected.
	
	According to such theory, the stress state acting on the tooth is
	\[ \sigma = \frac{F}{mb Y_l}\]
	where $b$ is the face width (figure \ref{fig:tooth}) and $Y_l$ is the Lewis factor taking into account for the shape of the tooth; such parameter depends on both the number of teeth $z$ and the pressure angle $\alpha$, and for the given data it evaluates to $Y_l = 0.308$.
	\begin{figure}[bt]
	\begin{subfigure}{.5\textwidth}
		\centering
		\includegraphics[scale=0.5]{Images/toothgear1.png}
		\caption{}
	\end{subfigure}%
	\begin{subfigure}{.5\textwidth}
		\centering
		\includegraphics[scale=0.6]{Images/toothgear2.png}
		\caption{}
	\end{subfigure}
	\caption{tooth details: section (a) and face width (b).}
	\label{fig:tooth}
	\end{figure}
	
	Given the ultimate tensile strength $\sigma_{uts} = 40MPa$ of plastic materials used in 3D printing machine and choosing a safety factor $\phi= 2$, reversing the Lewis equation gives us the minimum face width:
	\[ \sigma \leq \frac{\sigma_{uts}}{\phi} \qquad \Rightarrow \qquad b \geq \frac{F \phi}{Y_l m \sigma_{uts}} = 8.61mm \]
	thus a width $b = 10mm$ for the teeth has been chosen.
	
	\paragraph{Crushing verification} The gear is mounted to the motors shaft by means of a $3\times 3mm$ parallel key; the shaft is made of steel with high mechanical properties, while the gear, that can be 3D-printed, has lower stress capability, thus we need to verify the member at crushing.
	
	By the geometry of the gear reported in the annexes, the contact surface that is used to exchange the load is a rectangular surface of dimensions $b\times h = 10 mm \times 1.292mm$; given the augmented force $F = 447N$ (as the radius where the force is applied is reduced to $r = 4.25mm$) that has to be exchanged, the mean pressure at the contact surface is
	\[ p = \frac{F}{bh} = 34.62MPa \]
	Considering the allowable pressure $p_{all} = \sigma_{uts} = 40MPa$, then the safety factor against crushing of the gear is $\phi = 1.15$.
	
	\paragraph{Gear and rack design} With this calculation performed, using built-in tools provided by the software and simple sketches a custom design for both gear and rack has been developed in order to be 3D printed by the end user. Annex \ref{draw:gearrack} (page \pageref{draw:gearrack}) reports the technical drawing of both parts.
	
	Moreover exploiting the parametric capability of \texttt{AutoDesk Inventor} the rack length is fully modular and can be extended/reduced by a integer amount of teeth in order to match the 3D printer plate dimensions; in the drawing it has been considered a rack with $z = 60$ teeth giving an overall length $l = 377mm$. 
	
	
	